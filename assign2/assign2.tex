\documentclass{article}

\usepackage{amsmath}
\usepackage{amsthm}
\usepackage{graphicx}


\newtheorem{algorithm}{Algorithm}
\setlength\parindent{0pt}

\title{Group Assignment 2}
\author{Chance Zibolski, Dean Johnson}

\begin{document}
\maketitle

\section*{Algorithm 1}

\section*{Psuedocode}

\begin{verbatim}
Powerset(set):
  if set == {}:
      return {{}}
  else
    result = Powerset(set[1:])
    new_elements = []
    for x in Powerset(set[1:])
      new_elements += [set[0] + x]
    return result + new_elements
\end{verbatim}

\begin{verbatim}
Algorithm1(N, M, T, key_lockers, ball_lockers)
    // Every combination of keys
    key_sets = Powerset(key_lockers)
    // Create a list containing 0 if the locker has no ball, 1 if it does
    for i = 0, ..., T
        ball_index = ball_lockers[i]
        lockers[ball_index] = 1

    counts = []
    for key_set in key_sets:
        lockers = initialize list of size N to all 0's

        count = 0

        // The left most and right most lockers with keys are their own cases
        // We iterate from the edge locker with a key, keeping track of the farthest
        // ball we find in that direction. It's distance from the edge key locker
        // is the number of lockers we need to open for the edges.
        // We do this for both edges

        left_most_key = key_sets[0]
        right_most_key = key_sets[len(key_set)]

        // Setting last_ball_seen to the left_most_key means if we found no balls
        // then when we subtract, we get 0 lockers opened
        last_ball_seen = left_most_key
        for i = left_most_key down to 0
            if lockers[i] == 1:
                last_ball_seen = i
        count += left_most_key - last_ball_seen

        last_ball_seen = right_most_key
        for i = right_most_key, ..., N
              if lockers[i] == 1:
                  last_ball_seen = i
        count += last_ball_seen - right_most_key

        // Now we need to calculate the largest gap between two lockers with keys, or
        // the midpoint.  The number of lockers needed to be opened between those two
        // locker is the distance between the lockers minus the largest gap between a
        // ball and a locker or a ball and one of the keys.

        // We're using all the key's we're given
        count += len(key_set)

        // For the given combination of keys, compute how best to use it
        for i = 0; ..., len(key_set) - 1
            // This loops through pairs of keys
            left_key = key_set[i]
            right_key = key_set[i+1]

            biggest_gap = 0
            found = false
            left_index = left_key

            // Give a pair of keys, compute the biggest gap
            for j = left_key, ..., right_key + 1
                if lockers[j] == 1
                    gap = j - left_index
                    if gap > biggest_gap
                        biggest_gap = gap
                        found = true
                    left_index = j

            if found == true
                count += (right_key - left_key - biggest_gap)
        append count to counts
    return min(counts)
\end{verbatim}

\section*{Run-time analysis}
We create the powerset of keys, or every combination of keys, which takes
$\mathcal{O}(2^n)$ time.

We then, for every combination we determine the minimum number of lockers
that need to be opened to obtain all tennis balls. There are two loops, but
it takes $\mathcal{O}(n)$ time. Here's the explanation:

Given a set of keys, we're looping through the keys m times, where m is the
number of keys in the subset, and m <= n. 

\section*{Solutions}
\begin{enumerate}
\setlength{\itemsep}{1pt}
\item 11
\item 14
\item 7
\item 14
\item 18
\item 1
\item 15
\item 8
\end{enumerate}
\end{document}
